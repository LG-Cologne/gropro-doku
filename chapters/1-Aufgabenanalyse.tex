\chapter{Aufgabenanalyse}\label{ch:aufgabenanalyse}


\section{Interpretation der Aufgabe}\label{sec:interpretation-der-aufgabe}
Gefordert ist ein Programm, welches ein Gebiet übergeben bekommt und für dieses eine 100\%-Abdeckung von Funk mit möglichst wenigen Antennen errechnet.\\
Ein Gebiet hat dabei Höhenunterschiede und wird in 100mx100m Quadrate eingeteilt, wobei die Eckpunkte beziehungsweise deren Höhe in 100 m übergeben wird.
Die Höhe muss minimal 0 und maximal 6 betragen, die 6 steht hierbei für 600 m.
Eine Antenne ist 10 m hoch und kann auf Eckpunkten platziert werden.
Sie kann nicht durch Flächen hindurch senden, ist aber in ihrer Reichweite sonst unbegrenzt.
Gesucht wird die minimale Anzahl an Antennen und deren Position, um das gesamte Gebiet auszuleuchten.
Die Eingabe soll über Dateien erfolgen, welche die Größe eines Gebiets und dessen Höhenmeter, an den Eckpunkten der einzelnen Quadrate, enthalten muss und Kommentarzeilen enthalten kann.
Dabei werden nur Zahlen aus den natürlichen Zahlen $\mathbb{N}_0$ akzeptiert und unvollständige Eingaben zeilenweise mit dem letzten vorhanden Wert korrigiert beziehungsweise vervollständigt.
Für die Ausgabe ist zum einen die Ausgabe auf dem Bildschirm, in einer Datei und auch in einer Datei für Gnuplot zu berücksichtigen.
Sie soll die Beschreibung der Eingabe enthalten, die Größe des Gebiets in X und Y, die Anzahl der Antennen und deren Position.
Für Gnucobol zu erstellen sind die Dateien zur Definition, die Daten des Gebiets und für jede Antenne im jeweiligen Format.\\

Als Datenstruktur wird ein zwei-dimensionaler Float-Array genutzt, um einheitliche Rechenoperationen mit der 0.1 · 100 m hohen Antenne zu gewährleisten.
Bevorzugt wird float, da 32-Bit mehr als ausreichend für Rechenoperationen mit Zahlen von bis zu 6.1.
Zudem kann durch Hardwarebeschleunigung, oder auch ohne, eine kürzere Laufzeit erreicht werden, im Gegensatz zu Double-Operationen.\\

Die Eingabedatei wird zeilenweise eingelesen.
Beginnt eine Zeile mit einem Semicolon, wird diese als Kommentar beziehungsweise als Beschreibung interpretiert.
\begin{figure}[h]
    \centering
    \caption{Input-Restriktionen}
    \begin{itemize}[noitemsep]
        \item Eine Beschreibung sowie Kommentare sind optional.
        \item Die erste Nicht-Kommentar-Zeile muss die Größen des Gebiets enthalten.
        \item Die darauffolgenden Nicht-Kommentar-Zeilen müssen die Höhenangaben beinhalten.
        \item Nach einer Leerzeile dürfen keine anderen Zeilen, abgesehen von Kommentarzeilen, stehen.
        \item Pro Datei kann nur ein Gebiet eingelesen werden.
    \end{itemize}
    \label{fig:input-restrictions}
\end{figure}

Die Lösung des Problems wird mittels Brute-Force realisiert.
Es wird zuerst iterativ auf jeder möglichen Position nacheinander eine Antenne platziert und die Antenne ausgewählt, welche am meisten Felder abgedeckt hat.
Sind noch nicht alle Felder ausgeleuchtet, wird so lange wieder so vorgegangen, bis keine Felder mehr ohne Netzabdeckung existieren.
Es wird demnach in jedem Schritt greedy die beste Antenne gewählt.


\section{Fehlerarten}\label{sec:fehlerarten}
Die Eingabedatei kann verschiedene Integritätsbedingungen verletzen.
Das Programm muss diese Fehlerarten identifizieren und den Nutzer darüber informieren.

\subsection{Technische Fehler}\label{subsec:technische-fehler}
Wenn eine, vom Anwender ausgewählte, Datei nicht gefunden wird, bekommt er dies über eine FileNotFoundException von der Anwendung mitgeteilt.

\subsection{Syntaktische Fehler}\label{subsec:syntaktische-fehler}
Die Eingabedatei muss der Struktur aus~\nameref{fig:input-restrictions} entsprechen.
So kann zum Beispiel ein syntaktischer Fehler provoziert werden, indem die Höhendefinitionen vor der Gebietsgröße geschrieben werden.

\subsection{Semantische Fehler}\label{subsec:semantische-fehler}


\section{Fehlerbehandlung}\label{sec:fehlerbehandlung}

\subsection{Technische Fehler}\label{subsec:technische-fehler-behandlung}

\subsection{Syntaktische Fehler}\label{subsec:syntaktische-fehler-behandlung}

\subsection{Semantische Fehler}\label{subsec:semantische-fehler-behandlung}
\subsection{Sonderfälle}\label{subsec:sonderfaelle}
