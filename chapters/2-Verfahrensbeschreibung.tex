\chapter{Verfahrensbeschreibung}\label{ch:verfahrensbeschreibung}


\section{Gesamtsystem}\label{ver:sec:gesamtsystem}
Das System arbeitet nach dem EVA-Prinzpip. Die EVA-Segmente werden von einem Controller erweitert, welcher das Programm koordiniert und den Einstiegspunkt des Programms darstellt.

\subsection{Eingabe}\label{ver:subsec:eingabe}
Um ein vollständiges Bahnnetz abbilden zu können werden alle Bahnverbindungen aus der Eingabedatei eingelesen. Aus diesen Verbindungen können dann später die gewünschten Bahnstationen ermittelt werden. Um das Netz möglichst übersichtlich abzuspeichern werden wird Instanzen eines Objekt Zugverbindung (siehe \ref{ver:subsubsec:trainconnection}) erstellt welche dann weitergegeben werden können.\\

\subsection{Verarbeitung}\label{subsec:verarbeitung}
Die Liste mit den Zugverbindungen wird nun vom Controller entgegengenommen und kann nun weiterverarbeitet werden. Diese kann der Datenreduktion (siehe \ref{ver:subsubsec:reducer}) oder direkt dem Datenmodell (siehe \ref{ver:subsubsec:trainweb}) übergeben werden. \\ Das erstellte Datenmodell wird dann an den Algorithmus (siehe \ref{ver:subsec:berechnung}) übergeben. Dieser kann nun die minimale Anzahl und die Standorte der Servicestationen ermitteln. Das Ergebnis, bestehend aus einer Liste mit ermittelten Stationen wird dem Controller zurückgegeben.\\


\subsection{Ausgabe}\label{ver:subsec:ausgabe}
Das Abschließende Ergebnis wird nun an die Ausgabe weitergegeben. Diese nimmt die Liste mit den ermittelten Stationen entgegen und schreibt diese in eine Ausgabedatei.\\

\section{Datenstrukturen}\label{ver:subsec:datenstrukturen}
Um die Verarbeitung der Daten übersichtlich zu gestalten werden verschiedene Datenstrukturen erstellt. Dabei lässt sich unterscheiden in Speicher Strukturen und Verarbeitungsstrukturen.\\
\subsection{Speicherstrukturen}\label{ver:subsec:Speicherstrukturen}
Alle erstllten Datenstrukturen basieren auf ArrayListen. Diese bieten einerseits eine dynamische Speichermöglichkeit. Andererseits ermöglicht dies ein optimale Speicherung. Durch die Nutzung von ArrayListen können die Zugnetzdaten einmal gespeichert und dann als referenzen weitergegeben werden. Dies verhindert unnötige kopiererei und erleichtert Vergleiche.\\

\subsubsection{TrainConnection}\label{ver:subsubsec:trainconnection}
Hauptbestandteil der Datenvermittlung ist die Klasse TrainConnection. Diese besitzt als einziges Attribut eine ArrayList in welcher Stationen als String abgespeichert werden. Diese Stationen bilden dann eine vollständige  Zugverbindung ab.\\

Außerdem sind zwei Vergleichmethoden implementiert. Diese ermöglichen die Liste an Stationen miteinander zu vergleichen. Eine Methode vergleicht die Anzahl der Stationen. Die zweite Methode überprüft den Inhalt der Stationen\\

\subsubsection{Stations}\label{ver:subsubsec:stations}
Um im Verlaufe der Verfahren Strukturiert vorgehen zu können existiert die Klasse Stations. Diese speichert einen Namen als String und eine ArrayListe mit Bahnverbindungen(\ref{ver:subsubsec:trainconnection}) in der der die Station angefahren wird.

\subsubsection{TrainWeb}\label{ver:subsubsec:trainweb}
Mit diesen beiden Methoden lässt sich nun das gesamte Bahnnetz speichern. Dafür werden alle Bahnverbindungen die in der Eingabedatei eingelesen werden in einer ArrayList als Attribut in TrainWeb gespeichert. Aus dieser Liste werden alle Bahnstationen gefiltert. Für jede Bahnverbindung in der eine Bahnstation vorkommt wird die jeweilige Referenz auf die Verbindung in der Bahnstation gespeichert. Bei den Bahnverbindungen innerhalb der Stationen handelt es sich nicht um neue Objekte sondern die referenz zu der Uhrsprünglichen Verbindung.\\


% TODO Datenstruktur_Zusammenhänge.pdf hier einfügen


\subsection{Verarbeitungsstrukturen}\label{ver:subsec:verarbeitungsstrukturen}
Diese Speicherstrukturen werden in den folgenden Verarbeitungsstrukturen verwendet.\\
\subsubsection{Reducer}\label{ver:subsubsec:reducer}
Der Reducer nimmt die im Input erstellten Zugverbindungen entgegen und wendet die drei Reduktionsverfahren darauf an. Die ArrayListe an Zugverbindungen wird \\
\subsubsection{Algorithmus}\label{ver:subsubsec:algorithmus}

\section{Algorithmenbeschreibung}\label{ver:sec:verfahren}
\subsection{Berechnung der minimalen Anzahl an Servicestationen}\label{ver:subsec:berechnung}

\subsection{Beschreibung der drei Reduktionsverfahren}
\subsubsection{Doppelstationen}\label{ver:subsubsec:doppelstationen}
In diesem Verfahren werden alle Zugverbindungen nach Mehrfachvorkommen von Stationen geprüft. Sollte eine Station mehrfach in einer Verbindung angefahren werden, braucht diese nur einmal in der Zugverbindung gespeichert werden. Für die spätere Berechnung ist nur relevant welche Stationen in einer Zugverbindung angefahren werden. Die Reihenfolge der Stationen spielt somit keine Rolle. Somit gehen durch die einfache Speicherung von Stationen keine Inforamtionen verloren.\\

\subsubsection{Stationsabhängigkeiten}
In diesem Verfahren werden Stationsabhängigkeiten geprüft. Sollte in allen Zugverbindung die eine beliebige Station A beinhaltet, auch eine Station B angefahren werden kann die betrachtete Station A in allen Zugverbindungen entfernt werden. Es ist davon auszugehen, dass dabei kein Informationsverlust riskiert wird, da jede Zugverbindung der betrachteten Station A auch die gefundene Station B passieren wird. Die Menge der Zugverbindungen die Station A anfahren ist also eine Teilmenge der Verbindungen Station B und müssen nichtmehr gesondert gespeichert und überprüft werden. Dabei kann es dazu kommen, dass Zugverbindungen aus nurnoch einer Station bestehen.\\

\subsubsection{Implizite Zugverbindungen}
In diesem Verfahren wird überprüft ob eine Zugverbindung implizit in einer anderen Zugverbindung enthalten ist. Sollte dies der Fall sein, darf die größere Zugverbindung entfernt werden. Dies ist erlaubt, da alle Stationen der kleineren Verbindung auch in der größeren Verbindung angefahren werden. Eine Überprüfung der kleineren Zugverbindung ist also Ausreichend um beide Verbindung eine Servicestationen zu gewährleisten. Die größere Verbindung ist somit redundant.\\

