\chapter{Benutzeranleitung}\label{ch:benutzeranleitung}


\section{Vorbereiten des Systems}\label{sec:vorbereiten-des-systems}

\subsection{Systemvoraussetzungen}\label{subsec:systemvoraussetzungen}
Um sicherzustellen, dass das Programm lauffähig ist, sollte~\Betriebssystem~als Betriebssystem genutzt werden.
Es ist außerdem eine JRE oder JDK in der Version 17 oder höher vonnöten, um das Programm auszuführen.\\
Falls eine erneute Kompilierung des Programms gewünscht ist, empfiehlt es sich die JDK anstelle der JRE zu installieren.
Um diese JRE/JDK anschließend zu nutzen, muss das bin-Verzeichnis dieser in die PATH-Umgebungsvariable hinzugefügt werden.
\subsection{Installation}\label{subsec:installation}
Es ist keine Installation nötig, es reicht das~.zip-Verzeichnis zu entpacken.

\section{Programmaufruf}\label{sec:programmaufruf}
Nach dem Entpacken kann das Programm über den Befehl
\begin{center}
    \colorbox{gray!20}{
        \begin{minipage}{0.9\textwidth}
            java -jar GroPro-1.0.jar "Testbeispiele/Test1IHK.txt"
        \end{minipage}
    }
\end{center}
in der Eingabeaufforderung (CMD) oder beliebiger Bash ausgeführt werden.
\section{Testen der Beispiele}\label{sec:testen-der-beispiele}
Das Ausführen der automatischen Tests erfolgt über die Datei \enquote{RunTestBeispiele.cmd}.
In der Konsole wird nun das Programm für jede Datei im Ordner \enquote{Testbeispiele} ausgeführt.
Alle Dateiausgaben befinden sich im Anschluss im Ordner \enquote{Testbeispiele/out}.

\section{Kompilieren}\label{sec:kompilieren}
Zum Erzeugen der~.jar-Datei sollte Maven genutzt werden.
Der Einfachheit halber muss das bin-Verzeichnis der Maveninstallation ebenfalls in der PATH-Umgebungsvariable aufgenommen werden.
Anschließend lässt sich das Programm kompilieren, indem der Befehl
\begin{center}
    \colorbox{gray!20}{
        \begin{minipage}{0.9\textwidth}
            mvn package
        \end{minipage}
    }
\end{center}
im Root-Verzeichnis des Quellcodes (das ist der Ordner, indem sich die \enquote{pom.xml} befindet) ausgeführt wird.
Eine JAR-Datei befindet sich dann im Ordner \enquote{target}.