\documentclass[
    12pt, % Schriftgröße
    oneside, % zweiseitiger Modus
    ngerman, % deutsches Dokument
    BCOR=0mm, % Bindungskorrektur
    DIV=10 % Division (Anzahl Spalten/Zeilen pro Seite, bestimmt implizit Margins)
]{scrreprt}

\newcommand{\ArbeitTitelseite}{Dokumentation der Praktischen Arbeit\ zur Prüfung zum\ Mathematisch-technischen Softwareentwickler}
\newcommand{\titleDocument}{\ArbeitTitelseite}

\newcommand{\Autor}{Leon Jarosch}
\newcommand{\authorDocument}{\Autor}

\newcommand{\ArbeitKopfzeile}{Dokumentation der Praktischen Arbeit \year}
\newcommand{\ArbeitThema}{Bestimmung der minimal Anzahl von Service Station unter berücksichtiung verschiedener Reduktionsverfahren}
\newcommand{\Pruefungsnummer}{142 18723}
\newcommand{\Programmiersprache}{Java}
\newcommand{\JavaVersion}{19}
\newcommand{\Compiler}{Maven}
\newcommand{\Rechner}{MacBook Pro 2017}
\newcommand{\CPU}{2,3 GHz Dual-Core Intel Core i5}
\newcommand{\RAM}{8 GB 2133 MHz LPDDR3}
\newcommand{\Betriebssystem}{macOS 13.0.1 (22A400)}
\newcommand{\IDE}{Visual Studio Code}
\newcommand{\IDEVersion}{1.77.0}
\newcommand{\IDEURL}{https://code.visualstudio.com/}
\newcommand{\TexLiveVersion}{2023}
\newcommand{\Artifact}{groprosim}

\newcommand{\subjectDocument}{Großprogrammierung}
\newcommand{\locationDocument}{Köln}
\newcommand{\dateDocument}{\today}

\title{\titleDocument}
\author{\authorDocument}
\date{\dateDocument}

%Style importieren:
\usepackage{dokumentation}


\begin{document}
    % ============ Anfang =============
    % Titelseite
    \include{title}
    \begingroup
        \hypersetup{hidelinks}
        \tableofcontents
    \endgroup

    % =========== Zahlenteil ===========
    \chapter{Aufgabenanalyse}\label{ch:aufgabenanalyse}
Der Sachverhalt der Aufgabenstellung handelt von einem Bahnnetz. Dieses beinhaltet mehere Bahnstationen die Teil einer oder meherer Zugverbindungen sein können. Der Sachverhalt setzt vorraus, dass Züge bei jeder Fahrt an Servicestationen versorgt werden müssen.
\section{Interpretation der Aufgabe}\label{sec:interpretation-der-aufgabe}
Gefordert ist ein Programm, welches die Positionen von Servicestationen ermittelt. Dafür muss gelten, dass jede Zugverbindung des angegebenen Bahnnetz mindestens eine Servicestation beinhaltet. In der Folge sollen Bahnstationen ermittelt werden die für das Bahnnetz als Servicestationen dienen können und somit alle Zugverbindungen des Bahnnetzes abdecken. Die Anzahl der Servicestationen soll dabei so gering wie möglich sein.
\\
Um die Speicherkapazitäten zu minimieren und die Programmlaufzeit zu optimieren werden zusätzliche Reduktionsverfahren auf die Zugverbindungen angewendet werden.\\ 

Die Umsetzung erfolgt auf Basis eines Bahnnetzes, welches aus einer Datei eingelesen werden kann. Die Dateiendung ist dabei frei wählbar. Aus dieser Datei können die verschiedenen Zugverbindungen eingelesen werden. Diese bilden das Bahnnetz ab.\\
Beginnt eine Zeile mit einem \enquote{\#} so wird diese als Kommentar interpretiert. Einzelne Zugverbindungen werden mit Zeilenumbrüchen getrennt. Die darin enthaltenden Bahnstationen sind durch Semikolons seperierbar.\\

Auf diese Bahnstationen werden nun drei verschiedene Reduktionsverfahren angewendet. Diese sollen die Anzahl der Bahnstationen und Bahnverbindungen reduzieren, ohne dabei einen Informationsverlust zu riskieren.\\ 

Dafür werden zunächst alle Zugverbindungen nach Mehrfachvorkommen von Stationen geprüft. Sollte eine Station mehrfach in einer Verbindung angefahren werden, braucht diese nur einmal in der Zugverbindung gespeichert werden.

Als nächstes werden Stationsabhängigkeiten geprüft. Sollte in allen Zugverbindung die eine beliebige Station A beinhaltet, auch eine andere Station B auftauchen kann die betrachtete Station A entfernt werden. Es ist davon auszugehen, dass dabei keine Informationsverlust riskiert wird, da jede Zugverbindung der betrachteten Station A auch die gefundene Station B passiert. Ob die Stationen in jeder Verbindung gleich angefahren werden, spielt dafür keine Rolle.

Die letzte Reduktionstechnik überprüft ob eine Zugverbindung implizit in einer anderen Zugverbindung enthalten ist. Sollte dies der Fall sein, darf die größere Zugverbindung entfernt werden. Dies ist erlaubt, da alle Stationen der kleineren Verbindung auch in der größeren Verbindung angefahren werden. Die größere Verbindung ist somit redundant.\\
\\

Zur Berechnung des Algorithmuses werden die übergebliebenen Bahnstationen gesammelt. Anhand derer wird nun die minimale Anzahl an Servicestationen ermittelt. Der dafür zuständige Algorithmus ist ein iteratives Vorgehen, welches nach Greedy-Prinzipien arbeitet.\\
\\

Im Anschluss an die Berechnung wird die Ausgabe erzeugt. Diese besteht aus einer einzeiligen .txt Datei. In dieser werden, identisch zur Eingabedatei, die ermittelten Bahnstationen semikolongetrennt aufgelistet.\\

\begin{figure}[h]
    \centering
    \caption{Input-Restriktionen}
    \begin{itemize}[noitemsep]
        \item Es muss mindestens eine (nicht Kommentar-) Zeile vorhanden sein
        \item Jede Zeile muss mindestens 2 Stationen enthalten
        \item Jede Zeile darf beliebig viele Stationen enthalten
        \item Es dürfen beliebig viele Zeilen vorhanden sein
        \item Es dürfen beliebig viele Kommentar-Zeilen vorhanden sein
        \item Semikolons dürfen nicht am Anfang oder Ende einer Zeile stehen
        \item Semikolons dürfen nicht hintereinander stehen
    \end{itemize}
    \label{fig:input-restrictions}
\end{figure}

\begin{figure}[h]
    \centering
    \caption{Stationnamen-Restriktionen}
    \begin{itemize}[noitemsep]
        \item Ein Name darf nur aus Buchstaben von A bis Z bestehen
        \item Ein Name darf aus maximal 3 Buchstaben bestehen
    \end{itemize}
    \label{fig:stationname-restrictions}
\end{figure}


\section{Fehlerarten}\label{sec:fehlerarten}
Die Eingabedatei kann verschiedene Integritätsbedingungen verletzen.
Das Programm muss diese Fehlerarten identifizieren und den Nutzer darüber informieren.

\subsection{Technische Fehler}\label{subsec:technische-fehler}
Technische Fehler entstehen, wenn dem Programm ein nicht existierender Dateiname übergeben wird.

\subsection{Syntaktische Fehler}\label{subsec:syntaktische-fehler}
Die Eingabedatei muss der Struktur aus~\nameref{fig:input-restrictions} entsprechen.
So kann zum Beispiel ein syntaktischer Fehler provoziert werden, indem Stationnamen nicht semikolongetrennt sind.

\subsection{Semantische Fehler}\label{subsec:semantische-fehler}
Die eingelesenen Stationnamen müssen der Struktur aus~\nameref{fig:stationname-restrictions} entsprechen.
So kann zum Beispiel ein syntaktischer Fehler provoziert werden, indem Stationnamen Zahlen beinhalten.

\section{Fehlerbehandlung}\label{sec:fehlerbehandlung}

\subsection{Technische Fehler}\label{subsec:technische-fehler-behandlung}
Ist ein technischer Fehler vorhanden, wird eine Ausnahme geworfen und das Programm beendet.


\subsection{Syntaktische Fehler}\label{subsec:syntaktische-fehler-behandlung}
Wird ein syntaktischer Fehler identifiziert, wird eine Ausnahme geworfen und das Programm beendet. Außerdem wird eine Erklärung für die korrekte Syntax der Eingabedatei ausgegeben.

\subsection{Semantische Fehler}\label{subsec:semantische-fehler-behandlung}
Wird ein syntaktischer Fehler identifiziert, wird eine Ausnahme geworfen und das Programm beendet. Außerdem wird eine Erklärung für gültige Stationnamen ausgegeben.


\subsection{Sonderfälle}\label{subsec:sonderfaelle}
Nach der Datenreduktion kann es sein, dass eine Zugverbindung nurnoch aus einer Station besteht. Beim einlesen von Zugverbindungen ist dies ungültig und wirft eine Ausnahme. Nach dem Einlesen und somit auch nach der Datenreduktion ist dies jedoch wieder erlaubt und ist eine gültige Zugverbindung.\\

    \chapter{Verfahrensbeschreibung}\label{ch:verfahrensbeschreibung}


\section{Gesamtsystem}\label{sec:gesamtsystem}
Das System arbeitet nach dem EVA-Prinzpip. Die EVA-Segmente werden von einem Controller erweitert, welcher das Programm koordiniert und den Einstiegspunkt des Programms darstellt.

\subsection{Eingabe}\label{subsec:eingabe}

\subsection{Verarbeitung}\label{subsec:verarbeitung}
\subsection{Ausgabe}\label{subsec:ausgabe}

\section{Strukturen}\label{sec:strukturen}
\subsection{Datenstrukturen}\label{subsec:datenstrukt}
\subsubsection{TrainConnection}\label{subsubsec:trainconnection}
\subsubsection{Stations}\label{subsubsec:stations}
\subsubsection{TrainWeb}\label{subsubsec:trainweb}

\subsection{Beschreibung der drei Reduktionsverfahren}
\subsubsection{Doppelstationen}\label{subsubsec:doppelstationen}
In diesem Verfahren werden alle Zugverbindungen nach Mehrfachvorkommen von Stationen geprüft. Sollte eine Station mehrfach in einer Verbindung angefahren werden, braucht diese nur einmal in der Zugverbindung gespeichert werden. Für die spätere Berechnung ist nur relevant welche Stationen in einer Zugverbindung angefahren werden. Die Reihenfolge der Stationen spielt somit keine Rolle. Somit gehen durch die einfache Speicherung von Stationen keine Inforamtionen verloren.\\

\subsubsection{Stationsabhängigkeiten}
In diesem Verfahren werden Stationsabhängigkeiten geprüft. Sollte in allen Zugverbindung die eine beliebige Station A beinhaltet, auch eine Station B angefahren werden kann die betrachtete Station A in allen Zugverbindungen entfernt werden. Es ist davon auszugehen, dass dabei kein Informationsverlust riskiert wird, da jede Zugverbindung der betrachteten Station A auch die gefundene Station B passieren wird. Die Menge der Zugverbindungen die Station A anfahren ist also eine Teilmenge der Verbindungen Station B und müssen nichtmehr gesondert gespeichert und überprüft werden. Dabei kann es dazu kommen, dass Zugverbindungen aus nurnoch einer Station bestehen.\\

\subsubsection{Implizite Zugverbindungen}
In diesem Verfahren wird überprüft ob eine Zugverbindung implizit in einer anderen Zugverbindung enthalten ist. Sollte dies der Fall sein, darf die größere Zugverbindung entfernt werden. Dies ist erlaubt, da alle Stationen der kleineren Verbindung auch in der größeren Verbindung angefahren werden. Eine Überprüfung der kleineren Zugverbindung ist also Ausreichend um beide Verbindung eine Servicestationen zu gewährleisten. Die größere Verbindung ist somit redundant.\\


    \chapter{Programmbeschreibung}\label{ch:programmbeschreibung}

\section{Struktogramm}\label{sec:pap}

\section{Entwicklungsdokumentation}\label{sec:entwicklerdokumentation}
Die Dokumentation des Programms wurde in Javadoc vorgenommen und kann im Ordner javadoc eingesehen werden.
Hierzu kann die index.html aufgerufen werden
%    \addtocontents{toc}{\protect\newpage}
    \chapter{Testdokumentation}\label{ch:testdokumentation}

\section{Systemtests}\label{test:sec:systemtests}
Die Systemtests wurden mit Hilfe der Testdateien durchgeführt.\\
Im Folgenden wird auf die Systemtest mit den Testdateien eingegangen. Dazu wurden sich entsprechende Äquivalenzklassen überlegt und jeweils mit einem Vertreter dieser Klassen eine Testdatei erstellt und ein Test durchführt.\\
\subsection{Normalfälle}\label{test:sec:normalfaelle}
\begin{center}
    \includegraphics[width=\linewidth]{images/Tests/IHK-Beispiele/Beispiel1.png}
    \label{test:subsecpar:beispiel1}
\end{center}

\begin{center}
    \includegraphics[width=\linewidth]{images/Tests/IHK-Beispiele/Datenreduktion1.png}
    \label{test:subsecpar:Datenreduktion1}
\end{center}

\begin{center}
    \includegraphics[width=\linewidth]{images/Tests/IHK-Beispiele/Datenreduktion2.png}
    \label{test:subsecpar:Datenreduktion2}
\end{center}

\begin{center}
    \includegraphics[width=\linewidth]{images/Tests/IHK-Beispiele/Datenreduktion3.png}
    \label{test:subsecpar:Datenreduktion3}
\end{center}


\subsection{Fehlerfälle}\label{test:sec:fehlerfaelle}
Die folgenden Test decken die in \ref{auf:sec:fehlerarten} gennanten Fehlerfälle ab.

\subsubsection{Technische Fehler}\label{test:sec:technische-fehler}
\begin{center}
    \includegraphics[width=\linewidth]{images/Tests/Fehlerfälle/FileException.png}
    \label{test:subsecpar:einlese-fehler}
\end{center}

\begin{center}
    \includegraphics[width=\linewidth]{images/Tests/Fehlerfälle/LerreDatei.png}
    \label{test:subsecpar:leere-datei}
\end{center}

\subsubsection{Syntaktische Fehler}\label{test:sec:syntaktische-fehler}
\begin{center}
    \includegraphics[width=\linewidth]{images/Tests/Fehlerfälle/FormatException.png}
    \label{test:subsecpar:format-fehler}
\end{center}

\subsubsection{Semantische Fehler}\label{test:sec:semantische-fehler}
\begin{center}
    \includegraphics[width=\linewidth]{images/Tests/Fehlerfälle/InvalidNameException.png}
    \label{test:subsecpar:namen-sind-nicht-erlaubt}
\end{center}

\subsection{Grenzfälle}\label{test:sec:grenzfaelle}

\begin{center}
    \includegraphics[width=\linewidth]{images/Tests/Grenzfälle/einzigeStation.png}
    \label{test:subsecpar:einzige-station}
\end{center}

\begin{center}
    \includegraphics[width=\linewidth]{images/Tests/Grenzfälle/einzigeVerbindung.png}
    \label{test:subsecpar:einzige}
\end{center}


\subsection{Sonderfälle}\label{test:sec:sonderfaelle}

\begin{center}
    \includegraphics[width=\linewidth]{images/Tests/Sonderfälle/uerbergreifendeStation.png}
    \label{test:subsecpar:uerbergreifendeStation}
\end{center}

\begin{center}
    \includegraphics[width=\linewidth]{images/Tests/Sonderfälle/unabhaengigeStationen.png}
    \label{test:subsecpar:unabhaengigeStationen}
\end{center}


\section{Ausführliches Beispiel}\label{test:sec:ausfuehrliches-beispiel}
Im folgendem wird für ein Bahnnetz ein kompletter Algorithmen durchlauf beschrieben.\\

Dafür wird folgende Datei eingelesen:\\
\begin{center}
    \includegraphics[width=\linewidth]{images/Programmdurchlauf/eingabedatei.png}
    \label{test:subsecpar:eingabedatei}
\end{center}

Beim lesen der Datei werden Kommentare ignoriert und die Zeilen in Zugverbindungen umgewandelt. Dabei wird auf korrekte Syntax geachtet.
Sind alle Verbindungen eingelesen werden die Stationen ermittel. Zusammen werden diese in einem Zugnetz gespeichert. Das vorliegende Zugnetz liegt folgendermaßen vor:\\

\begin{center}
    \includegraphics[width=\linewidth]{images/Programmdurchlauf/Datenstruktur01.png}
    \label{test:subsecpar:datenstruktur1}
\end{center}
\\
Diese Zugnetz wird nun Reduziert. Dafür werden die drei Reduktionsverfahren angewandt.\\
Die erste Reduktiontechnik überprüft ob eine Station mehrfach in einer Zugverbindung vorkommt. Ist dies der Fall werden alle doppelten vorkommen entfernt.\\
Da in diesem Fall keine doppelten Stationen vorkommen, wird das Netz nicht verändert.\\
\\
Die zweite Reduktionstechnik fasst Stationen zusammen, die keinen Aussagewert h aben. Dies ist dann der Fall, wenn jede Verbindung die Station A anfährt auch eine Station B beinhaltet. In diesem Fall kann Station A gelöscht werden.\\
Stationen \texttt{K}, \texttt{S}, \texttt{B}, \texttt{L} und \texttt{DA} tauchen in jeweils nur einer Verbindung vor. Solang diese noch weiter Stationen beinhalten können diese sofort entfernt werden. Betrachtet man Station \texttt{M} und \texttt{N} lässt sich feststellen, dass diese in allen Verbindungen auch \texttt{FFM} anfahren und somit ebenfalls entfernt werden können.\\
Das Bahnnetz liegt nun folgendermaßen vor:\\

\begin{center}
    \includegraphics[width=\linewidth]{images/Programmdurchlauf/Datenstruktur02.png}
    \label{test:subsecpar:datenstruktur2}
\end{center}
\\
Die ∏dritte Reduktionstechnik überprüft ob eine Zugverbindung implizit in einer anderen Zugverbindung enthalten ist. Ist dies der Fall kann die größere Zugverbindung entfernt werden. Das gilt auch für Verbindungen die die genau selben Stationen anfahren.\\
Dafür werden die Verbindungen mit den wenigsten Stationen betrachtet. In diesem Fall bestehen Verbindung 2 und 5 nurnoch aus der Station \texttt{FFM}. Verbindungen, welche die Teilstrecke \texttt{FFM} beinhalten können nun entfernt werden. Das heißt auch, das Verbindungen die equivalent zur betrachteten Verbindung ist, also die exakt selben Stationen anfährt, ebenfalls entfernt werden kann.\\
Die übergebliebenen Verbindungen ergeben folgendes Bahnnetz:\\

\begin{center}
    \includegraphics[width=\linewidth]{images/Programmdurchlauf/Datenstruktur03.png}
    \label{test:subsecpar:datenstruktur3}
\end{center}
\\
Das nun reduzierte Bahnnetz kann jetzt dem Algorithmus übergeben werden, welcher die Stationen ausgewählt, die als Servicestation verwerndet werden können.\\
Dafür wir die Station ausgewählt, die in den meisten Verbindungen vorkommt. In diesem Fall haben alle Stationen gleich viele Verbindungen sodass eine beliebige Station gewählt werden kann. Diese Station wird in eine Liste von Stationen gegeben.\\

\begin{center}
    \includegraphics[width=\linewidth]{images/Programmdurchlauf/Datenstruktur04.png}
    \label{test:subsecpar:datenstruktur04}
\end{center}
\\
Nun wird verglichen, ob es noch Verbindungen gibt, die noch keine Servicestation beinhalten. Dazu werden alle Verbindungen der Stationen in der neuen Liste gesammelt und um anschluss mit allen Verbindungen des Netzplans zu verglichen.\\
Die hier hinzugefügte Station \texttt{HH} deckt die Verbindung 1 ab. Verbindung 2 ist noch nicht versorgt, sodass nun weitergesucht werden muss\\
Dafür werden alle noch nicht versorgten Verbindungen betrachtet und die Stationen ausgewählt, die in den meisten Verbindungen vorkommt.\\
Da es sich nurnoch im eine Verbindung handelt kann man eine beliebige Station der Verbindung auswählen. So kann auch \texttt{FFM} an die Liste angehängt werden.\\

\begin{center}
    \includegraphics[width=\linewidth]{images/Programmdurchlauf/Datenstruktur05.png}
    \label{test:subsecpar:datenstruktur05}
\end{center}
\\
Bevor überprüft werden kann, ob alle Verbindungen versorgt sind, wird jetzt überprüft, ob alle Stationen der Liste wirklich gebraucht werden.\\
Dazu wird verglichen ob die komplette Liste die selben Verbindungen abdecken kann wie die Liste ohne eine Station. Sollte dies der Fall sein, wird die Station aus der Liste entfernt. Wurden mehere Stationen gefunden werden, wird die Station gelöscht, die in mehr Verbindungen vorkommt entfernt. Konnte eine Station gelöscht werden wiederholt sich der Kontrollprozess solange, wie noch Stationen gefunden werden, die entfernt werden können.\\
In diesem Fall gibt es jedoch keine Verbindung, die gelöscht werden kann.\\

Jetzt kann überprüft werden, ob alle Verbindungen versorgt sind. In diesem Fall beinhalten alle Verbindungen mindestens eine Station der neuen Liste.\\
Somit ist die Liste der Servicestationen vollständig und kann ausgegeben werden.\\

Dafür wird folgendes Dokument ausgegeben:\\
\begin{center}
    \includegraphics[width=\linewidth]{images/Programmdurchlauf/ausgabedatei.png}
    \label{test:subsecpar:ausgabedatei}
\end{center}
\\
Der Programmdurchlauf ist nun abgeschlossen.\\
    \chapter{Zusammenfassung und Ausblick}\label{ch:zusammenfassung-und-ausblick}


\section{Zusammenfassung}\label{sec:zusammenfassung}
Es wurde ein Programm entwickelt, welches die Berechnung von Servicestationen wie gefordert umsetzt.
Eingabe, Verarbeitung und Ausgabe laufen in voneinander unabhängigen Prozessen. Beliebige Datensätze können dem Programm übergeben. Die Verarbeitung erfolgt in zwei Schritten in dem zunächst eine optionale Reduktion auf die Eingabedaten durchgeführt wird und anschließend die eigentliche Berechnung stattfindet. Das Programm endet, wenn das gefundene Ergebnis ausgegeben wurde.\\
Die angesprochenen Teilaufgaben wurden in einzelne Klassen aufgeteilt. Diese können unabhängig voneinander verändert und erweitert werden.
Die Verfahren und ihre Details wurden ausführlich dokumentiert. Die Wahl von unterschiedlichen Datenstrukturen wurde begründet und der Programmentwurf ist durch diverse UML- und Nassi-Shneiderman-Diagramme gut nachvollziehbar. Für Entwickler wurde eine Dokumentation direkt im Source-Code erstellt.

\section{Ausblick}\label{sec:ausblick}
\subsection{Einführung von Pattern}\label{sec:pattern}
Wie bereits angesprochenen sind alle Teilaufgaben systematisch getrennt. Hier bietet es sich an ein Stategie-Pattern zu implementieren. Dieses würde es ermöglichen die einzelnen Teilaufgaben austauschbar zu machen. Außerdem ist ein Rahmen gegeben sodass Veränderungen keinen Einfluss auf die Benutzung der einzelnen Komponenten haben können.\\

\subsection{Algorithmus}\label{sec:algorithmus}
Der implementierte Algorithmus glänz mit Laufzeiteffizienz. Er ist in der Lage ein Bahnnetz mit 10.000 Verbindungen und 500 Staitionen in bis zu 5:11:29 Minuten auszurechnen. Je nach Hardware kann dies varieren.\\
Beim Vergleich mit anderen Algorithmen ist jedoch aufgefallen, dass nicht immer die minimale Anzahl an Servicestationen ermittelt wird. Dies stellt logischerweise ein Problem dar und sollte dringend ausgebessert werden.\\

Dafür bieten sich folgende Ansätze an. \\

\subsubsection{Reiner Brute-Force Ansatz}\label{sec:brute-force}
Um sicher zu gehen, immer ein minimales Ergebnis zu erhalten, kann ein Brute-Force Ansatz implementiert werden. Dieser würde alle möglichen Kombinationen an Lösungsstationen durchgehen und die Lösung mit der geringsten Anzahl an Stationen ausgeben. Auf unkomplexe Bahnnetze können bei diesem Ansatz gute Ergebnisse erzielt werden. Da es sich beim Brute-Force Ansatz jedoch um ein expondenzielles Verfahren handelt, ist die Laufzeit bei komplexeren Bahnnetzen nicht wünschenswert.\\
Dies gilt auch wenn ein Abbruchkriterium eingebaut wird, welches neue Ergebnisse durch das aktuell beste Ergebnis einschränkt. Damit werden nurnoch Kombinationen überprüft, die eine Verbesserung zum aktuellen Ergebnis ermöglichen können. Auch hier müssen jedoch Laufzeiteinbussen in kauf genommen werden\\

\subsubsection{Erweiterung durch Brute-Force}
Um von der Laufzeiteffizienz des aktuellen Algorithmus zu profitieren und trotzdem ein tatsächlich minimales Ergebnis zu erhalten, kann der Algorithmus um einen Brute-Force Ansatz erweitert werden. Dafür wird das Ergebnis der des Greedy-Algorithmus mit einem Brute-Force Ansatz validiert.\\
Dazu werden wird die Liste an Servicestationen überprüft und für jede möglichen Stations Kombination ermittelt. Diese werden dann ebenfalls mit allen Kombinationen von allen verfügbaren Stationen verglichen. Bedingung ist das die Auswahl an Bahnstation kleiner ist als die der aktuellen Service Stationen. Sollten beide Kombinationen die selben Verbindungen abdecken, wird die neue Kombination als mit den betrachteten Servicestationen getauscht.\\
Sollten keine Tauschmöglichkeiten mehr gefunden werden ist davon auszugehen eine tatsächlich minimal Anzahl an Staionen gefunden zu haben.\\
Auch dieser Ansatz ist expondenziell. Durch eine bereits approximierte Lösung ist jedoch davon auszugehen, dass weniger Vergleiche getätigt werden müssen und die Laufzeit besser wird.\\

\subsection{Testabdeckung}\label{sec:testabdeckung}
Die hier dokumentierten Test, testen nur einen erfolgreichen oder auch fehlerhaften Durchlauf des Programms. Da die Projektstruktur eine saubere Trennung der einzelnen Komponenten gewährleistet, bietet es sich an diese mit Unit-Tests zu testen.\\
    % ============= Buchstabenteil ==============
    \renewcommand{\thechapter}{\Alph{chapter}}%
    \setcounter{chapter}{0}
    \chapter{Abweichung und Ergänzung}\label{ch:abweichung-und-ergaenzung}

\section{Probleme}
Im Verlaufe der Implemeniterung ist aufgefallen, dass manche Projektabschnitte in der Klausur falsch verstanden wurden. In der folge dessen wurden falsche schlüße gezogen.\\

\subsection{Datenreduktionsverfahren 2}
Ein Beispiel ist die zweite Reduktionsmethode. Diese entfernt nicht nur, wie zunächst angenommen, benachtbarte Stationen. Die Reihenfolge der Stationen ist bei der Reduktion hinfällig. Dadurch gibt es massive unterschiede in der implementierung der zweiten Redukitonsmethode.\\

\subsection{Algorithmus}
Innerhalb der Klausur sind gedanklich die Stationen und Verbindungen miteinander verschmolzen. Teilweise werden Verfahren dadurch falsch gefolgert und falsch dargestellt.


\section{Datenstrukturen}
Um ein Übersichtliches System zu schaffen, wurden in der Strukturierung auf mögliche Hilfsklassen verzichtet. Damit sollte die leserlichkeit erhalten bleiben und die Struktur Übersichtlich gestalten.\\
Dieser Ansatz wurde aus mehreren Gründen verworfen.\\
\subsection{Trennung von Speicher und Verarbeitung}
Die geplante Modell Klasse sollte ursprünglich nicht nur die Daten speichern sondern ebenfalls die Reduktionsverfahren wie auch den Algorithmus berechnen. Dies führte zu Übersichts Problemen. Erster Ansatz war nun das Datenmodell und die Verfahren zu trennen.\\ 

\subsection{Trennung der Verfahren}
Dadurch konnten Klassen deutlich reduziert werden. Die gewünschte Übersicht war immernoch nicht gegeben. Dies führte zur jetzigen Trennung.\\

\subsection{Datenmodell}
Ebenso musste festgestellt werden, dass das Datenmodell nicht optimal gewählt wurde. Ursprünglich sollten Stationen nur innerhalb der Verbindungen gespeichert werden. Dafür war die verwendung von geschachtelten \texttt{Arrays} angedacht.\\
Aus meheren gründen wurde diese Idee verworfen.\\
Fehlende dynamik des Datentypen, fehlende Übersichtlichkeit der Verbindungen wie auch ein anbahnen von dauerhaften kopieren von Stationen haben zu einem Umdenken geführt.\\
\\
Die Speicherklasse Bahnhof (\ref{ver:subsubsec:station}) wurde in der Folge um die Klasse Bahnverbindung (\ref{ver:subsubsec:trainconnection}) ergänzt.\\
Außerdem wurden sämtliche \texttt{Arrays} durch \texttt{ArrayListen} ersetzt.\\
\\
Dies änderungen führen zu der beschriebenen Datenstruktur (\ref{ver:fig:datenstruktur_zusammenhaenge}).\\
    \include{chapters/B-Benutzeranleitung}
    \chapter{Entwicklungsumgebung}\label{ch:entwicklungsumgebung}
\begin{table}[ht]
    \centering
    \label{tab:environment}
    \begin{tabular}{p{3.5cm}p{9cm}}
        \textbf{Betriebssystem} & \Betriebssystem\\
        \textbf{Hardware} & \Rechner~\CPU~\RAM \ RAM\\
        \textbf{Compiler} & Java Development-Kit \JavaVersion\\
    \end{tabular}
\end{table}
    \chapter{Verwendete Hilfsmittel}\label{ch:verwendete-hilfsmittel}

\begin{itemize}
    \item \IDE \ \IDEVersion \\ Entwicklungsumgebung und Editor für Java und andere Programmiersprachen \\\url{\IDEURL}
    \item Maven\\Build-Tool für Java\\\url{https://maven.apache.org}
    \item TexLive \TexLiveVersion\\Softwarepaket für \LaTeX\\\url{https://miktex.org/}
    \item Visual Paradigm\\Programm zur Modellierung von Software-(Diagrammen) \\\url{https://www.visual-paradigm.com/}
    \item Git\\Versionsverwaltungssystem \\\url{https://git-scm.com/}
\end{itemize}
    \include{chapters/E-Erklaerung}
    \include{chapters/F-Aufgabenstellung}
\end{document}